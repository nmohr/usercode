\section{Signal}\label{sec:signal}

The CMS minimal supergravity low-mass benchmark points
have been designed to cover 
different decay modes of the neutralinos within supersymmetry. 
The mass spectra of the benchmark points have been calculated using the Softsusy code~\cite{softsusy}. 
All branching ratios have been calculated with the SUSYHit program~\cite{susyhit} 
and the events are simulated using Pythia~\cite{pythia}. 
The k-factor for the cross section at 7~TeV is calculated using a modified version of Prospino~2~\cite{prospino}. 
In mSUGRA observable signal is produced strongly followed be (very) long decay chains 
leading to several hard jets (at least two). 
The escaping neutralino leads to missing transverse energy. 
This fact allows to define a search region to observe an excess over the SM 
and is used as main event selection as described in Section~\ref{sec:eventselection}.
It consits of
\begin{itemize}
\item Two opposite sign same flavour leptons within acceptance of $p_T>10$~GeV
\footnote{In the first version of the note the lepton $p_T$ cut was at 5~GeV, which gives a better
sensitivity to low mass mSugra. The electrons are not fully commissioned to 5~GeV and since there
was no excess observed, we took an conservative approach to raise the threshold for this year and to continue
commissioning the electron identification at low $p_T$, such that we can use low $p_T$ thresholds in the next year.}
and  $|\eta|<2.4$.
\item High $\HT>350~GeV$ with at least to jets to be as model independent as possible.
\item High $\MET>150~GeV$, which is set as such that we expect roughly one di-leptonic $t\bar{t}$ event in this years dataset.
\end{itemize}

%\begin{figure}[hbtp]
%\begin{center}
%\subfigure[]{\label{fig:yieldOSByChannelAndBMPoint}\includegraphics[angle=0,width=0.49\textwidth]{diLepton342_combinedOSpromptPromptTauCount}}\hfill
%\end{center}
%     \subfigure[]{\label{fig:yieldSSByChannelAndBMPoint}\includegraphics[angle=0,width=0.49\textwidth]{diLepton342_combinedSSpromptPromptTauCount}}\hfill
% \caption{Expected event yield for a given benchmark point and exclusive di-lepton channel with respect to the total expected di-lepton
%    event yield at that benchmark point. Only the hadronic decay of the $\tau$-lepton is counted in the ``Tau'' channels, as
%    this is the distinction that is made on detector level.\label{fig:yieldByChannelAndBMPoint}}
%\end{figure}

The number of expected events in a given di-lepton channel varies with the mSUGRA point realised by nature. 
%Thus, itsimportance cannot be judged from a single point in mSUGRA phase-space. 
A range of benchmark points has been studied to anticipate different possible signal constellations and not
tune towards a specific set of parameters. 
Figure~\ref{fig:yieldByChannelAndBMPoint} shows the fraction of di-leptonic events splitted by flavour
for a di-lepton selection on generator level. In all studied benchmark points
the flavour correlated production dominates over the uncorrelated production.


