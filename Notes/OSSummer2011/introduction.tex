\section{Disclaimer}

Please note that most of the methods used in this analysis have also been documented in SUS-08-004,
SUS-09-002 (AN-2009/83), AN-2010/167 and AN-2010/373.

\section{Introduction}

The standard model of particle physics (SM) leads to a number of unsolved issues like the 
hierarchy problem and it provides no solution for pressing questions arising from astrophysical 
observations, most notably dark matter. In Supersymmetry (SUSY) a natural candidate for dark matter 
can be found if R-parity conservation is assumed. Supersymmetric particles (sparticles) have not 
been observed up to now which implies that they have to be heavy. 
On the other hand to provide a solution to the hierarchy problem their masses have to be in the TeV range.
These prejudices lead to a signature of (many) hard jets and large missing transverse energy.

%The long anticipated and succesfull start of the Large Hadron Collider (LHC) with a center of mass energy of 7~TeV 
%allows us to explore this new TeV range very early on. 
Of special interest are robust signatures in leptonic final states which can be probed with the CMS experiment.
If R-parity is conserved the lightest neutralino escapes detection and no mass peaks can be observed 
in SUSY decay chains. A key point after discovery will be the determination of the sparticle properties. 

The purpose of this analysis is to observe or exclude a significant excess of di-leptons 
over the various backgrounds.
The dataset consists of 204~\pbi of proton-proton collisions collected by CMS during LHC
running in 2011.

In Sec.~\ref{sec:signal} we define the signal benchmark points, that are used in this analysis.
Section~\ref{sec:physobj} describes the technical details of the object selection
and in Sec.~\ref{sec:eff} we discuss lepton efficiencies used in the background
prediction.
Section~\ref{sec:fakes} describes a method to determine the contribution
of non-prompt leptons in the final event selection.
In Sec.~\label{sec:eventselection} we define the signal regions, including a discussion of the main
Standard Model backgrounds and their yields. %Here we also discuss the trigger and its efficiency.
Section~\ref{sec:ofossubtraction} deals with the background prediction methods,
which are used to predict the number of Standard Model backgrounds in the signal
regions. The results are presented in Sec.~\ref{sec:results}. 
Finally we set a limit and conclude.
