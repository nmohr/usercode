\section{Background prediction methos}

\subsection{Different flavour subtraction} \label{sec:ofossubtraction}
Since the signal region is expected to be absolutely 
dominated by $t\bar{t}$-production we use
the opposite flavour subtraction to predict
backgrounds where uncorrelated leptons are being produced.
It relies only on the knowledge of the ratio
of electron to muon reconstruction efficiency $r_{e\mu}$,
which we derive in Sec.~\ref{sec:eff}.

Under the assumption of lepton unversality
The following two formulas hold for any background
where di-leptons are being produced uncorrelated 
(e.g. top-pairs events, $Z \rightarrow \tau^{+} \tau{-} \rightarrow 
l^{+} \l^-$, $WW$-production):
$$
n_{ee} = \frac12n_{e\mu}r_{\mu{}e}, \quad n_{\mu\mu} = \frac12\frac{n_{e\mu}}{r_{\mu{}e}}.
$$
A closure test of the method has been performed
using a simulated top-pair sample and
we observe a good agreement between prediction and MC truth:
$$
n_{ee} = 16.9 \pm 2.8 (\textnormal{stat.}) \; \textnormal{ (16.2 MC)}, \quad n_{\mu\mu} = 19.8 \pm 3.3 (\textnormal{stat.}) \; \textnormal{ (21.5 MC)}.
$$


\subsection{Control Region}

To gain confidence in the performance of this technique we wish to test it on a high statistic sample, dominated by 
$t\bar{t}$, as is our expected signal region is. To do this, we are required to use the leptonic trigger sample, 
described in~\cite{avi} including the lepton $p_T$ thresholds of 20 (10)~GeV, 
since we need to lower the $H_T$ requirement inherent in the hadronic trigger sample. Using 
this sample our control region is defined by:  $100 < H_T < 350$~GeV and $\MET>80$~GeV,
where we expect almost no Z+Jets contribution anymore and are dominated by $t\bar{t}$.
The control region is disjunct from the signal region, however it also suffers,
from potential signal contamination, in case of a very soft \MET spectrum of the new physics.
The distribution of the events in the invariant mass distribution split by lepton flavour combinations
is shown in Fig.~\ref{fig:control_InvMass}.

In this region we observe $26$ opposite flavour opposite sign candidates and
subtract $1.0\pm0.5$ predicted fake events. Therefore we obtain $25.0\pm0.5$ $t\bar{t}$-
like candidate events, which we use to obtain a prediction for the same flavour combinations
using the ratio of Tab.~\label{tab:tnp_eff}.

Table~\ref{tab:OFexpect} shows the number of expected SM background events in the control region $100 < H_T < 350$~GeV 
and $\MET>80$~GeV, as well as the prediction from the background estimation techniques, for an integrated
luminosity of 35~pb${}^{-1}$. We observe a good agreement between prediction and observation.

\begin{table}[hbt]
\begin{center}
\caption{\label{tab:OFexpect}Number of predicted and observed events in the control region, defined as: $100 < H_T < 350$~GeV and $\met > 80$~GeV.}
\begin{tabular}{l|cc}
\hline
                       & \multicolumn{2}{c}{Control Region}               \\
\hline 
Process                & $ee$          & $\mu\mu$        \\
\hline
$t\bar{t}$ from $e\mu$ & $11.7\pm 2.4$ & $13.4\pm 2.8$   \\
Fake leptons           & $0.5\pm 0.3$  & $0.4\pm0.2$                  \\
\hline
Total predicted        & $12.2\pm 2.4$ & $13.8 \pm 2.8$  \\
\hline\hline
Total observed         & $10$          & $15$          \\
\hline \hline
SM MC         & $8.4\pm 0.2$  & $10.5 \pm 0.3$    \\
LM0                    &  $3.7\pm0.2$  & $4.2\pm0.2$     \\
LM1                    &  $0.5\pm0.1$  & $0.7\pm0.1$     \\

\hline
\end{tabular}
\end{center}
\end{table}


