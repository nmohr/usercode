\section{Systematic uncertainties}\label{sec:systematics}

Systematic uncerteinties arise from uncertainties of the event selection
expected in simulation compared to the performance from the actual detector.
A global normalisation uncertainty comes from an uncertainty on the total 
integrated luminosity of 11\%~\cite{lumiCMS}.

\subsection{Systematic uncertainty for the lepton selection}

We do observe a good agreement from tag and probe within 2\%.
In simulation the difference of the lepton
selection between Z+jets and top-pair events is within 7\%,
which we use as a systematic uncertainty.
The inclusive lepton efficiency of leptons in the mSUGRA
benchmark points differ by 25\% to the efficiency measured
at the Z resonace. 
However this drop in efficiency is simulated by the MC.
We take 10\% uncertainty on the leptonic part as a systematic uncertainty in the
limit setting procedure.

\subsection{Pile-up}

Since the lepton efficiency is measured including pile-up, we  include the effect
and do not correct for the difference in lepton selection efficiency.
Nevertheless pile-up has some effect on the jet and missing transvers energy 
selection. It has been studied in simulation and found to be negligible for the high \HT
and \MET selection.
Therefore no additional uncertainty is assigned to account for
pile-up.
For leptons above 10~GeV the effect of pile-up in the isolation is below 3\%
per lepton and therefore we do not assign an additional systematic uncertainty.

\subsection{Uncertainty on the hadronic selection}

The uncertainty on the jet-energy-scale is smaller than 5\% for
particle flow jets~\cite{jetMETUncertainty}. This uncertainty
directly translates into a scale uncertainty on the \HT selection.
For particle flow MET and uncertainty on the scale of 5\% is assumed~\cite{pfMETUNcertainty}.
We vary both fully correlated and find that it changes the acceptance
at LM0 by 12\%, while the change at LM1 is smaller (5\%).
We take this as systematic uncertainty in the limit setting method.
The uncertainty in the $t\bar{t}$-sample is found to be 30\%
in the final signal selection, while it is smaller in the leptonic control region (7\%). 
This enters the predicted number from simulation.

\subsection{Uncertainty on the fake background prediction}

For the fake-component of the background a uncertainty of 50\% is assumed,
which is described in detail in Sec.~\ref{sec:fakes}.

The main uncertainty for the opposite flavour subtraction arises from
the uncertainty on the lepton selection in a hadronic event environment.
While the inclusive efficiency for leptons originating from the Z-Boson
can be determined with a systematic uncertainty of 2\%, it cannot be 
determined with that precision in the kinematic regime of the search.
Therefore the difference between the efficiency from the Z and $t\bar{t}$
is assumed as a systematic uncertainty (5\%).

The systematic uncertainties on the yields from simulation are
summarised in Tab.~\ref{tab:systematics}.

\begin{table}[hbtp]
\caption{Summary of the systematic uncertainties for \label{tab:systematics}}
\begin{center}
\begin{tabular}{|l||c|c|c|} \hline
Systematic uncertainty    &   $t\bar{t}$   &   LM0        & LM1\\\hline \hline
Cross-section &   39\% & -    &- \\\hline  
\HT + \MET &   30\% & 12\%    &5\% \\\hline  
Lepton &   10\% & 15\%    &15\% \\\hline  
Luminosity &   11\% & 11\%    &11\% \\\hline\hline
Total &   51\% & 22\%    &19\% \\\hline  
\end{tabular}
\end{center}
\end{table}

